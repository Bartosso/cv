% !TEX TS-program = lualatex

\documentclass[a4paper,11pt]{article}

\usepackage[usenames,dvipsnames]{xcolor}
\usepackage{libertine}
\usepackage{fontawesome5}
\usepackage{longtable}
\usepackage[cm]{fullpage}

% headers and footers
\usepackage{fancyhdr}
\usepackage{lastpage}
\pagestyle{fancy}
\fancyfoot[L]{Yury Egorov}
\fancyfoot[C]{Page \thepage\ of \pageref*{LastPage}}
\fancyfoot[R]{\today}
\renewcommand{\footrulewidth}{0.4pt}% default is 0pt
\fancyhead{}
\renewcommand{\headrulewidth}{0pt}

% StackOverflow-like tags
% https://tex.stackexchange.com/a/311949/142692
% https://tex.stackexchange.com/questions/387499/how-to-create-a-border-that-looks-like-a-tag
\usepackage{tikz}
\definecolor{tagbg}{RGB}{225,236,244}
\definecolor{tagtxt}{RGB}{88,115,159}
\newcommand{\sotag}[1]{\tikz[baseline]{\node[anchor=base, rounded corners=0.5ex, text height=1.5ex, text depth=.25ex, fill=tagbg, draw=tagbg, text=tagtxt] {#1};}}

% Helpers for adding job entries
\newcommand{\job}[2]{\large\sffamily \textbf{#1} at \textbf{#2}}
\newcommand{\sep}{\multicolumn{2}{c}{}\\}

% tweak the url colors
\usepackage{hyperref}
\definecolor{linkcolor}{rgb}{0,0.2,0.6}
\hypersetup{colorlinks,breaklinks,urlcolor=linkcolor, linkcolor=linkcolor}

% nicer-looking section titles
\usepackage{titlesec}
\titleformat{\section}{\Large\scshape\raggedright}{}{0em}{}[\titlerule]
\titlespacing{\section}{0pt}{1em}{3pt}

\begin{document}

% --------------------TITLE-------------
\par{\centering
		{\Huge \textsc{Yury Egorov}
	}\bigskip\par}

\hrule
\vspace{0.5em}
\begin{tabular}{rl}
  \textsc{Phone:}     & +7 (906) 825-57-49\\
  \textsc{Email:}     & \href{mailto:yurij.a.egorov@gmail.com}{yurij.a.egorov@gmail.com}\\
  \textsc{Socials:}   & \faLinkedin{} \href{https://www.linkedin.com/in/yuryegorov/}{www.linkedin.com/in/yuryegorov/}
                      | \faGithub{} \href{https://github.com/yordwynn}{github.com/yordwynn}
                      | \faTelegram{} \href{https://t.me/yordwynn}{t.me/yordwynn}
\end{tabular}

\section{Summary}
\begin{tabular}{p{0.9\textwidth}}
  I'm a software developer and PhD student with deep learning in computer vision and research background (Mathematical Modeling, mumerical Methods and Software Systems). Currently diving into Scala and functional programming. Remote is preferable\\\\

  Interests: \sotag{functional-programming} \sotag{scala} \sotag{cats} \sotag{deep-learning} \sotag{computer-vision} \sotag{algorithms} 
\end{tabular}

\section{Programming Proficiency}
\begin{tabular}{rl}
  \textsc{Languages:}& Scala, Python, Java, C\#, C++, Erlang\\
  \textsc{Libraries:}& NumPy, PyTorch, Cats-Effect, Circe\\
  \textsc{Tools:}& IntelliJ SDK, Docker, MongoDB, PostgreSQL\\
\end{tabular}

\section{Work Experience}
\begin{longtable}{r|p{0.72\textwidth}}
  \textsc{Sep 2018--present} & \job{Teaching Assistant, PhD student of Software Department}{University of Tyumen}, Russia \\
    &\sotag{research} \sotag{computer-vision} \sotag{algorithms} \sotag{discrete-math} \sotag{python} \sotag{pytorch}\\&\\
    &I work on the research project dedicated to the human abnormal actions recongition on video. Main responsibilities are:
    \begin{itemize}
      \item recognition models developement
      \item experiments conduction
      \item scientific papers preparation
      \item hosting laboratory classes mostly in algorithms and desrcete math
    \end{itemize}\\\sep
  
  \hline
  \multicolumn{2}{r}{\footnotesize\itshape (abbreviated work history below)}\\\sep
  
  \textsc{Jan 2019--June 2019} & \job{Software Developer (contractor)} {Concord Soft}, Russia \\(6 months)
    &\sotag{java} \sotag{spring-boot} \sotag{python} \sotag{mongodb} \sotag{docker}\\
    &\begin{itemize}
      \item developed the geological formats convertion service
      \item supported the database client for computational tasks metadata management
    \end{itemize}\\\sep

  \multicolumn{2}{r}{\footnotesize\itshape (cont. on the next page)}\\\sep
  \newpage
  
  \textsc{Apl 2017--May 2018} & \job{Software Engineer}{Baspro}, Russia \\(1 year, 2 month)
    &\sotag{geological-data-processing} \sotag{c++} \sotag{mfc} \sotag{postgresql}\\
    &\begin{itemize}
      \item developement of the kern data storage and processing module
    \end{itemize}\\\sep
  
  \textsc{Nov 2014--Dec 2015} & \job{Software Developer}{University of Tyumen}, Russia \\(1 year, 2 months)
       \\{Jan 2016--Apl 2017} & \job{Head Software Developer}{University of Tyumen}, Russia \\(1 year, 4 months)
    &\sotag{java-script} \sotag{bootstrap} \sotag{sass} \sotag{c\#} \sotag{cisco-webex} \sotag{rabbitmq} \sotag{postgresql} \sotag{asp.net}\\
    &\begin{itemize}
      \item developed the web service for remote education as fronted developer mostly
      \item performed Cisco Webex integration
    \end{itemize}\\\sep

\end{longtable}

\section{Specializations}
\begin{tabular}{rl}
  \textsc{Courses:}
  &CS231n: Convolutional Neural Networks for Visual Recognition | Stanford (took yourself)\\
  &Introduction to Scala (Russian) | Stepik (\href{https://stepik.org/cert/204169}{certificate})
\end{tabular}

\end{document}
